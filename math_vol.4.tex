\documentclass[b5paper,uplatex,dvipdfmx,fleqn]{jsarticle}

\usepackage[top=1.5cm, bottom=2cm, left=2cm, right=2cm]{geometry}
\usepackage{bxpapersize}
\usepackage{fancybox}
\usepackage{tcolorbox}
\tcbuselibrary{raster,skins}
\usepackage{otf}
\usepackage{amsthm}
\usepackage{amsmath, amssymb,ascmac}
\usepackage{tikz}
\usetikzlibrary{calc}
\usepackage{mathtools}
\usepackage{enumerate}
\usepackage{framed, color}
\usepackage{multicol}
\usepackage{tcolorbox}
\tcbuselibrary{raster,skins}


\definecolor{shadecolor}{gray}{0.80}

\theoremstyle{definition}
\newtheorem{thm}{定理}[section]
\newtheorem{lem}[thm]{補題}
\newtheorem{prop}[thm]{命題}
\newtheorem{cor}[thm]{系}
\newtheorem{ass}[thm]{仮定}
\newtheorem{conj}[thm]{予想}
\newtheorem{dfn}[thm]{定義}
\newtheorem{rem}[thm]{注}
\newtheorem{fprb}[thm]{例題}
\newtheorem{prb}[thm]{問題}
\newtheorem*{eg}{例}
\newtheorem*{pf}{証明}
\newtheorem*{sol}{解答}
\newtheorem*{rem*}{注}


%%tcolorboxの定義
\newcounter{reidaibangou}
\newtcolorbox{reidai}[1][]{enhanced,boxrule=0.5mm,
    top=2pt,left=44pt,right=4pt,bottom=2pt,arc=0mm,
    colframe=black!30!gray,
    boxrule=1pt,
    underlay={
    \node[inner sep=1pt,black!50!black]at ([xshift=22pt,yshift=-10pt]interior.north west) {\stepcounter{reidaibangou}\bfseries\gtfamily 例題\thereidaibangou};},
    segmentation code={%
    \draw[dashed] (segmentation.west)--(segmentation.east);
    \node[inner sep=1pt,black!50!black] at ([xshift=22pt,yshift=-9pt]segmentation.south west) {\bfseries\gtfamily 解答};},
    before upper={\setlength{\parindent}{1zw}},
    before lower={\setlength{\parindent}{1zw}},
}


\renewcommand{\labelenumi}{(\theenumi)\ }

\renewcommand{\thepart}{\arabic{part}}

\renewcommand{\baselinestretch}{1.3}

\renewcommand{\thesubsection}{問題\arabic{subsection}}

\begin{document}

\renewcommand{\postpartname}{講}

\part{指数関数}
\noindent
\hrule height 0.5pt depth 0.5pt width 10cm
\vspace{0.5cm}
\noindent
\colorbox{black}{\textcolor{white}{\textgt{\large 基本事項}}}\vspace{0.4cm}\\
\textgt{\large \ajvarDiamond\ 指数法則}\vspace{0.2cm}\\
\setcounter{section}{1}
\textgt{❶ 指数法則}\vspace{0.2cm}

「文字 $a$ を $n$ 個掛け合わせたもの」を \textgt{$\boldsymbol{a}$ の $\boldsymbol{n}$ 乗} といい,$\boldsymbol{a^{n}}$ と書く(特に,$n = 1$ のとき $a^{1}$ を省略して $a$ と書く).ここで,$a^{n}$ の $n$ を 累乗の\textgt{指数} といい,$\boldsymbol{a}$ を \textgt{底} という.このときの $a$ は 正数とは限らず,$0$ や 負数でも良い(複素数でも成り立つ).

今まで,中等学校以前で学習した累乗の指数は「正の整数」であったが,これを「任意の整数」に拡張することができる.細かい説明を除けば,次の \textgt{指数法則} が成り立つ.\vspace{0.2cm}

\begin{tcolorbox}[title=\textgt{指数法則},sharp corners]
$m,\; n$ は整数で,$ab \neq 0$ とする.このとき,次の指数法則が成り立つ.
\begin{enumerate}
\item
$\boldsymbol{a^{m}a^{n} = a^{m + n}}$
\item
$\boldsymbol{(a^{m})^{n} = a^{mn}}$
\item
$\boldsymbol{(ab)^{n} = a^{n}b^{n}}$
\end{enumerate}
\end{tcolorbox}\vspace{0.2cm}

特に,$k$ が正整数で $a \neq 0$ のとき,$\displaystyle \boldsymbol{a^{-k} = \frac{1}{a^{k}}}$ と定義されることは注意されたい.\\

この法則から,様々な「指数計算の定理」を導くことができる.例えば
\[
\left( \frac{b}{a} \right)^{n} = \frac{b^{n}}{a^{n}}\;\;\; (a \neq 0)
\]
となる.これを証明してみよう.

指数法則の (3),(2) を用いると,
\[
\left( \frac{b}{a} \right)^{n} = (ba^{-1})^{n} = b^{n}(a^{-1})^{n} = b^{n}a^{-n} = \frac{b^{n}}{a^{n}}
\]
のようになる.

\newpage
\noindent
\textgt{❷ 累乗根}\vspace{0.2cm}

$n$ を正整数とするとき,「$n$ 乗して $a$ になる数」を \textgt{$\boldsymbol{a}$ の $\boldsymbol{n}$ 乗根} といい,$\boldsymbol{\sqrt[n]{a}}$ と書く.$2$ 乗根,$3$ 乗根,$4$ 乗根,$\cdots$ をまとめて \textgt{累乗根} という.以下に累乗根の性質をまとめる.\vspace{0.2cm}

\begin{tcolorbox}[title=\textgt{累乗根の性質},sharp corners]
$a \geqq 0,\; b \geqq 0,\; m,\; n$ を正整数とするとき,次が成り立つ.
\begin{enumerate}
\item
$\displaystyle \boldsymbol{\sqrt[m]{a}\sqrt[m]{b} = \sqrt[m]{ab}}$\vspace{0.2cm}
\item
$\displaystyle \boldsymbol{\frac{\sqrt[m]{a}}{\sqrt[m]{b}} = \sqrt[m]{\mathstrut \frac{\,a\,}{b}},\;\; (b \neq 0)}$\vspace{0.2cm}
\item
$\boldsymbol{(\sqrt[n]{a})^{m} = \sqrt[n]{a^{m}}}$\vspace{0.2cm}
\item
$\boldsymbol{\sqrt[m]{\sqrt[n]{a}} = \sqrt[mn]{a}}$
\end{enumerate}
\end{tcolorbox}\vspace{0.2cm}

\noindent
\textgt{❸ 指数の有理数への拡張}\vspace{0.2cm}

「指数」を「有理数」まで拡張して,意味を持たせることができる.例えば,$a^{\frac{1}{3}}$ などである.

\begin{tcolorbox}[title=\textgt{分数(有理数)指数の定義},sharp corners]
$a >0$ で,$m$ を任意の整数,$n$ を任意の正整数とするとき
\[
\boldsymbol{a^{\frac{m}{n}} = \sqrt[n]{a^{m}}}
\]
である.
\end{tcolorbox}\vspace{0.2cm}

\noindent
\textgt{❹ 指数の大小関係}\vspace{0.2cm}

指数の大小関係の同値性をまとめる.

$a > 1$ のとき,
\[
\boldsymbol{a^{x} < a^{y}\;\; \Leftrightarrow\;\; x < y}
\]

$0 < a < 1$ のとき,
\[
\boldsymbol{a^{x} < a^{y}\;\; \Leftrightarrow\;\; x > y}
\]

\newpage
「数の分類」をすると
\[
\textgt{実数} \begin{cases}
\textgt{有理数} \cdots \text{分数で表される数}\\
\textgt{無理数} \cdots \text{分数では表されない数}
\end{cases}
\]
であるが,実は無理数も指数に取り入れることができ,例えば
\[
1 < \sqrt{3} < 2
\]
であるから
\[
2 < 2^{\sqrt{3}} < 2^{2} = 4
\]
などと評価できる.\vspace{1cm}


\begin{reidai}
次の式を計算し,簡単にせよ.
\begin{multicols}{3}
\begin{enumerate}
\item
$\sqrt[3]{5} \div \sqrt[12]{5} \times \sqrt[8]{25}$
\item
$\sqrt{6} \times \sqrt[4]{54} \div \sqrt[4]{6}$
\item
$(3^{-2} \times 9^{\frac{2}{3}})^{\frac{3}{2}}$
\end{enumerate}
\end{multicols}\vspace{0.2cm}

\tcblower


\begin{align*}
(1)\hspace{0.5cm}\sqrt[3]{5} \div \sqrt[12]{5} \times \sqrt[8]{25}
&= 5^{\frac{1}{3}} \div 5^\frac{1}{12} \times 5^\frac{1}{4}\;\;\; (\because\, \sqrt[8]{25} = \sqrt[8]{5^{2}}) \\
&= 5^{\frac{1}{3} - \frac{1}{12} + \frac{1}{4}} \\
&= 5^{\frac{1}{2}} = \boldsymbol{\sqrt{5}}.
\end{align*}

\begin{align*}
(2)\hspace{0.5cm}\sqrt{6} \times \sqrt[4]{54} \div \sqrt[4]{6} 
&= \sqrt{6} \times \sqrt[4]{3^{2}} \cdot \sqrt[4]{6} \div \sqrt[4]{6} \\
&= 6^{\frac{1}{2}} \times 3^{\frac{1}{2}} \cdot 6^{\frac{1}{4}} \div 6^{\frac{1}{4}} \\
&= 6^{\frac{1}{2}} \cdot 3^{\frac{1}{2}} = \boldsymbol{\sqrt{18}}.
\end{align*}

\begin{align*}
(3)\hspace{0.5cm}(3^{-2} \times 9^{\frac{2}{3}})^{\frac{3}{2}}
&= 3^{-3} \times 3^{2} \\[10pt]
&= \boldsymbol{\frac{1}{\,3\,}}.
\end{align*}

\vspace{0cm}
\end{reidai}


\newpage
\noindent
\textgt{\large \ajvarDiamond\ 指数関数}\vspace{0.2cm}\\
\setcounter{section}{2}
\setcounter{thm}{0}
\noindent
\textgt{❺ 指数関数のグラフ}\vspace{0.2cm}

$a$ が $1$ でない正数であるとき,関数
\[
\boldsymbol{y = a^{x}}\;\;\;\; (a > 0,\; a \neq 1)
\]
を \textgt{$\boldsymbol{a}$ を底とする指数関数} という.指数関数のグラフの概形は次の図のようになる.\\

\begin{tikzpicture}
  \draw [-stealth] (-3,0) -- (3,0)node[right]{$x$};
  \draw [-stealth] (0, -1) -- (0,4.5)node[left]{$y$};
  \draw [thick] plot[domain=-3:2.15] (\x, {pow(2,\x)});
  \draw (0,0)node[below left]{O};
  \draw [dashed] (1,0)node[below]{$1$} -- (1,2) -- (0,2)node[left]{$a$}; 
  \draw [dashed] (2,0)node[below]{$2$} -- (2,4) -- (0,4)node[left]{$a^{2}$}; 
  \draw [dashed] (-1,0)node[below]{$-1$} -- (-1,1/2) -- (0,1/2)node[right]{$\dfrac{1}{\,a\,}$}; 
  \draw (-0.7,3)node[left]{$\boldsymbol{a > 1}$ \textgt{の場合}};
  \draw (0,1)node[above left]{$1$};
\end{tikzpicture}\ \ \ \ 
\begin{tikzpicture}
  \draw [-stealth] (-3,0) -- (3,0)node[right]{$x$};
  \draw [-stealth] (0, -1) -- (0,4.5)node[left]{$y$};
  \draw [thick] plot[domain=-2.15:3] (\x, {pow(1/2,\x)});
  \draw (0,0)node[below left]{O};
  \draw [dashed] (-1,0)node[below]{$-1$} -- (-1,2) -- (0,2)node[right]{$\dfrac{1}{\,a\,}$}; 
  \draw [dashed] (-2,0)node[below]{$-2$} -- (-2,4) -- (0,4)node[right]{$\dfrac{1}{\,a^{2}\,}$}; 
  \draw [dashed] (1,0)node[below]{$1$} -- (1,1/2) -- (0,1/2)node[left]{$a$}; 
  \draw (0.7,3)node[right]{$\boldsymbol{0 < a < 1}$ \textgt{の場合}};
  \draw (0,1)node[above right]{$1$};
\end{tikzpicture}\vspace{0.2cm}

\begin{tcolorbox}[title=\textgt{指数関数のグラフの特徴},sharp corners]
$y = a^{x} := f(x) \;\; (a > 0,\; a \neq 1)$ のグラフは,
\begin{enumerate}
\item
定点 $(0, 1)$ を通る.
\item
$x$ 軸の上側にある.(常に $y > 0$ である)
\item
$a > 1$ のとき増加で,$0 < a < 1$ のとき減少である.\vspace{0.2cm}
\item[($*$)\ ]
$y = a^{x}$ と $y = (a^{-1})^{x} = \dfrac{1}{\,a^{x}\,}$ は $y$ 軸対称.\vspace{0.1cm}\\
実は上の図において,左は $a = 2$,右は $a = \dfrac{1}{\,2\,}$ のときのグラフである.
\end{enumerate}
\end{tcolorbox}

\begin{tcolorbox}
\textgt{補 グラフの平行移動}\\
\indent
曲線 $y = f(x)$ を $x$ 軸方向に $a$,$y$ 軸方向に $b$ だけ平行移動して得られる曲線の方程式は,
\[
y - b = f(x - a).
\]
\end{tcolorbox}

\begin{reidai}
次のグラフを書け.
\begin{multicols}{2}
\begin{enumerate}
\item
$y = \left( \dfrac{\,1\,}{3} \right)^{x}$\vspace{0.2cm}
\item
$y = 2 \cdot 3^{x} - 1$
\end{enumerate}
\end{multicols}\vspace{0.2cm}

\tcblower

\begin{enumerate}
\item \phantom{a}\\
\begin{tikzpicture}
  \draw [-stealth] (-3,0) -- (3,0)node[right]{$x$};
  \draw [-stealth] (0, -1) -- (0,5.3)node[left]{$y$};
  \draw [thick] plot[domain=-1.5:3] (\x, {pow(1/3,\x)});
  \draw (0,0)node[below left]{O};
  \draw [dashed] (-1,0)node[below]{$-1$} -- (-1,3) -- (0,3)node[right]{$3$};
  \draw [dashed] (1,0)node[below]{$1$} -- (1,1/3) -- (0,1/3)node[above=5, left=2]{$\dfrac{1}{\,3\,}$};
  \draw (0,1)node[above right]{$1$};
\end{tikzpicture}
\item \phantom{a}\\
\begin{tikzpicture}
  \draw [-stealth] (-3,0) -- (3,0)node[right]{$x$};
  \draw [-stealth] (0, -2) -- (0,6)node[left]{$y$};
  \draw [thick] plot[domain=-3:1.13] (\x, {2*pow(3,\x) - 1});
  \draw (0,0)node[above right]{O};
  \draw [dashed] (1,0)node[below]{$1$} -- (1,5) -- (0,5)node[left]{$5$}; 
  \draw [dashed] (-1,0)node[above]{$-1$} -- (-1,-1/3) -- (0,-1/3)node[below=5, right=2]{$-\dfrac{1}{\,3\,}$};
  \draw (0,1)node[above left]{$1$};
  \draw [dashed] (-3,-1) -- (3,-1);
  \draw (0,-1)node[below right]{$-1$};
\end{tikzpicture}
\end{enumerate}

\vspace{0cm}
\end{reidai}

\newpage
\noindent
\textgt{\large \ajvarDiamond\ 指数方程式・不等式}\vspace{0.2cm}\\
\setcounter{section}{1}
\textgt{❻ 指数方程式}\vspace{0.2cm}

指数方程式の基本系は,$\boldsymbol{f(a^{x}) = 0}$ の形のものである.これは $a^{x} = X$ と置き換えることによって
\[
f(X) = 0\;\;\; (X > 0)
\]
を解くことに帰着する.不等式も同様に置き換えて解けば良い.以下実例で確認しよう.

\begin{reidai}
次の方程式を解け.

(1)\ \ 
$2^{3x + 2} - 4^{x} + 2^{x + 1} - 5 = 0$ \hspace{2cm}
(2)\ \ 
$\displaystyle \begin{cases}
2^{x} + 2^{y} = 40 \\
2^{x + y} = 256
\end{cases}$

\tcblower

\noindent
(1)\ \ $2^{x}$ を $X (>0)$ とおく.
\[
2^{3x + 2} = 4 \cdot (2^{x})^{3} = 4X^{3},\;\; 4^{x} = (2^{x})^{2} = X^{2},\;\; 2^{x + 1} = 2 \cdot 2^{x} = 2X
\]
であるから,与式は
\[
4X^{3} - X^{2} + 2X - 5 = 0\;\; \Leftrightarrow\;\; (X - 1)(4X^{2}  +3X + 5) = 0
\]
である.$X > 0$ であるから,常に $4X^{2} + 3X + 5 > 0$ が成り立つので
\[
X = 2^{x} = 1.\;\;\; \therefore\; \boldsymbol{x = 0}.
\]

\noindent
(2)\ \ $2^{x} = X,\; 2^{y} = Y$ とおくと,$2^{x + y} = 2^{x} \cdot 2^{y} = XY$ であるから
\[
\begin{cases}
X + Y = 40 \\
XY = 256
\end{cases}.
\]

$X,\; Y$ は $2$ 次方程式 $t^{2} - 40t + 256 = (t - 8)(t - 32) = 0$ の根であるから
\[
\begin{cases}
X = 2^{x} = 8 \\
Y = 2^{x} = 32
\end{cases} \text{または}\;\;\; \begin{cases}
X = 32 \\
Y = 8
\end{cases}.
\]

\[
\therefore\; \boldsymbol{\left(\begin{array}{c} x \\ y \end{array}\right) = \left(\begin{array}{c} 3 \\ 5 \end{array}\right) \textgt{または} \left(\begin{array}{c} 5 \\ 3 \end{array}\right)}.
\]

\vspace{0.21cm}

\end{reidai}

\begin{reidai}
次の不等式を解け.\vspace{0.2cm}

\noindent
(1)
$2^{x} + 2^{-x} < \dfrac{\,17\,}{4}$\hspace{2cm}
(2)
$\left( \dfrac{\,1\,}{2} \right)^{2x} < \left( \dfrac{\,1\,}{2} \right)^{x^{2}}$\vspace{0.4cm}\\
(3)
$8^{x + 1} - 4^{x + \frac{3}{2}} + 2^{x + 1}(1 - 2^{x}) < 0$\vspace{0.2cm}

\tcblower

\noindent
(1)\ \ 
$2^{x} = X (> 0)$ とおくと,$2^{-x} = \dfrac{1}{\,X\,}$ であるから,所与の式は
\[
X + \frac{\,1\,}{X} < \frac{\,17\,}{4}
\]
となり,$X > 0$ であるから分母を払い整理すると
\[
4X^{2} - 17X + 4 = (4X - 1)(X - 4) < 0.
\]
\[
\therefore\; \frac{\,1\,}{4} < X < 4.\;\;\;\;\;\; \therefore\; \boldsymbol{-2 < x < 2}.
\]

\vspace{0.4cm}

\noindent
(2)\ \ 
$\left( \dfrac{\,1\,}{2} \right)^{x}$ は 単調に減少する関数であるから,
\[
2x > x^{2}\;\; \Leftrightarrow\;\; x(x - 2) < 0.
\]
\[
\therefore\; \boldsymbol{0 < x < 2}.
\]

\vspace{0.4cm}

\noindent
(3)\ \ 
$2^{x} = X$ とおくと,
\[
8^{x + 1} = 8X^{3},\;\; 4^{x + \frac{3}{2}} = 8X^{2},\;\; 2^{x + 1} = 2X
\]
であるから,所与の式は
\[
8X^{3} - 8X^{2} + 2X(1 - X) < 0\;\; \Leftrightarrow\;\; 8X^{3} - 10X^{2} + 2X < 0.
\]
すなわち,
\[
2X(X - 1)(4X - 1) < 0.
\]
$X > 0$ より,
\[
\frac{\,1\,}{4} < X < 1.\;\;\; \therefore\; \boldsymbol{-2 < x < 0}
\]
である.
\vspace{0.55cm}

\end{reidai}

\newpage
\noindent
\colorbox{black}{\textcolor{white}{\textgt{\large 演習問題}}}
\begin{prb}
方程式
\[
4^{x} - 3 \cdot 2^{x} + 3 \cdot 2^{-x} + 4^{-x} = 0
\]
について,次の問いに答えなさい.
\begin{enumerate}
\item
$X = 2^{x} - 2^{-x}$ とおくとき,上の方程式を $X$ の式で表しなさい.
\item
上の方程式の実数解 $x$ をすべて求めなさい.
\end{enumerate}
\end{prb}\vfill

\begin{prb}
方程式
\[
2^{x} - (\sqrt{2})^{x + 1} - 4 = 0
\]
を解け.
\end{prb}\vfill

\begin{prb}
実数 $k$ を定数とする.$x$ と $y$ に関する連立方程式
\[
\begin{cases}
2^{x + 1} + 3^{y} = 2 \\
k \cdot 2^{x} - 3^{y} = 3k - 1
\end{cases}
\]
の解が存在するような $k$ の値の範囲は $\frac{\;\;\;\fbox{\;\; \textbf{シ}\;\; }\fbox{\;\; \textbf{ス}\;\; }\;\;\;}{\;\;\;\fbox{\;\; \textbf{セ}\;\; }\;\;\;} < k < \frac{\;\;\;\fbox{\;\; \textbf{ソ}\;\; }\;\;\;}{\;\;\;\fbox{\;\; \textbf{タ}\;\; }\;\;\;}$ である.
\end{prb}\vfill

\newpage
\begin{prb}
$0 < x < 2$ を満たす実数 $x$ に関する $2$ つの条件
\begin{align*}
&p : 2 \left( \frac{\,1\,}{4} \right)^{2x - 1} - 9 \left( \frac{\,1\,}{4} \right)^{x} + 1 < 0 \\
&q : \cos \left( \frac{\pi}{2x + 1} \right) \left\{ \cos \left( \frac{\pi}{2x + 1} \right) - \frac{\,1\,}{2} \right\} \left\{ \cos \left( \frac{\pi}{2x + 1} + 2 \right) \right\} < 0
\end{align*}
について,以下の問いに答えよ.

\begin{enumerate}
\item
条件 $p$ を満たす $x$ の範囲を求めよ.
\item
条件 $q$ を満たす $x$ の範囲を求めよ.
\item
命題 $p \Rightarrow q$ の真偽を調べよ.また,命題 $p \Rightarrow q$ の裏を述べ,その真偽を調べよ.
\end{enumerate}
\end{prb}\vfill

\begin{prb}
$y > 0$ とするとき,不等式
\[
y^{\frac{2}{x}} + y^{-\frac{2}{x}} - 6(y^{\frac{1}{x}} + y^{-\frac{1}{x}}) + 10 \leqq 0
\]
について次の各問に答えよ.
\begin{enumerate}
\item
$X = y^{\frac{1}{x}} + y^{-\frac{1}{x}}$ とするとき,この不等式を $X$ を用いて表せ.
\item
この不等式を満たす点 $(x, y)$ の全体が表す図形を座標平面上に図示せよ.
\end{enumerate}
\end{prb}\vfill

\begin{prb}
$f(x) = 16 \cdot 9^{x} - 4 \cdot 3^{x + 2} - 3^{-x + 2} + 9^{-x}$ とし,$t = 4 \cdot 3^{x} + 3^{-x}$ とおくとき,以下の問い答えよ.
\begin{enumerate}
\item
$t$ の最小値とそのときの $x$ の値を求めよ.
\item
$f(x)$ を $t$ の式で表せ.
\item
$x$ の方程式 $f(x) = k$ の相異なる実数解の個数が $3$ 個であるとき,定数 $k$ の値と,$3$ つの実数解を求めよ.
\end{enumerate}
\end{prb}\vfill


\end{document}